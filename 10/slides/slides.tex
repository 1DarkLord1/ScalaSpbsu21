\documentclass[aspectratio=169]{beamer}

% SETUP =====================================
\usepackage[T1,T2A]{fontenc}
\usepackage[utf8]{inputenc}
\usepackage[russian]{babel}
\usepackage{listings}
\usepackage{array}
\usepackage{amssymb}
\usepackage{pifont}
\usepackage{minted}
\usepackage{pgf-pie}
\usepackage{fancyvrb}
\usepackage{../../beamerthemeslidesgeneric}
% SETUP =====================================

\title{Software Testing}
\author{Artyom Semyonov}
\institute{СПБгУ, СП}
\date{\today}

\begin{document}

\frame{\titlepage}

% definition
\begin{frame}{Software testing}
  \begin{block}{What is software testing?}
    \textbf{Software testing} is an investigation conducted to provide stakeholders
    with information about the quality of the software product or service under test.
  \end{block}
\end{frame}

% only automated
\begin{frame}{Manual vs Automated}
  Who does testing?
  \begin{itemize}
    \item \textbf{Manual testing}~-- perfomed by a \textit{human}.
    \item \textbf{Automated testing}~-- perfomed by a \textit{program}.
  \end{itemize}
  \pause
  Why does manual testing still exist?
  \pause
  \begin{itemize}
    \item $t$~-- time to manually perform some test scenario once
    \item $T$~-- time to automate the scenario
    \item $n$~-- count of the scenario repeat
    \item $T >> t$ (in most cases)
    \item if $T > n * t$ then manual testing is better.
  \end{itemize}
  \bigskip
  \pause
  This lecture is \underline{about \textbf{Automated testing} only}.
\end{frame}

% only functional
\begin{frame}{Functional vs non-functional}
  Software testing is about an application behaviour.
  \begin{itemize}
    \item \textbf{Functional testing}~-- check if \textit{functionality} is correct.
    \item \textbf{Non-functional testing}~-- other checks.
      \begin{itemize}
        \item Scalability testing
        \item Performance testing
        \item Security testing
        \item Usability testing
        \item \ldots
      \end{itemize}
  \end{itemize}
  \bigskip
  \pause
  This lecture is \underline{about \textbf{Functional testing} only}.
\end{frame}

% levels
\begin{frame}{Testing levels (1)}
  We can perform testing of the application on different levels.
  \begin{itemize}
    \item \textbf{Unit testing}
      \begin{itemize}
        \item Example: functions, classses
        \item Unfair testing
        \item But fast, simple and clear, stable
      \end{itemize}
    \pause
    \item \textbf{Integration testing}
      \begin{itemize}
        \item Example: interaction with other system
        \item More fair testing
        \item Slower, more complicated, not so stable
      \end{itemize}
    \pause
    \item \textbf{System testing}
      \begin{itemize}
        \item Example: use the server API from a \textit{different machine}
        \item The fairest testing
        \item Slow, complex, unstable
        \item \ldots
      \end{itemize}
  \end{itemize}
\end{frame}

% levels nbs
\begin{frame}{Testing levels (2)}
  Testing levels are sometimes vague.
  \begin{itemize}
    \item Testing of backend API~-- Integration? System?
    \item Testing of DAO~-- Unit? Integration?
  \end{itemize}
  \pause
  \bigskip
  What you should know:
  \begin{itemize}
    \item Which test is fair and which is less fair.
    \item Best testing practices which are universal for any testing level.
          We will discuss them.
  \end{itemize}
\end{frame}

% prefere blackbox
\begin{frame}{The <<box>> approach}
  \begin{itemize}
    \item \textbf{Black-box testing}~-- The tester knows \textit{nothing} about the implementation
    \item \textbf{White-box testing}~-- The tester knows \textit{everything} about the implementation
    \item \textbf{Grey-box testing}~-- The tester knows \textit{something} about the implmenetation
  \end{itemize}
  \bigskip
  \pause
  <<Knows>> means <<As if he knows>>.
  Prefer the \textbf{Black-box} approach by following the encapsulation principle.
\end{frame}

\begin{frame}{Terminology}
  <<Test>> has a vague meaning. It's better to use more precise terms:
  \begin{itemize}
    \item \textbf{Test-case}~-- a test scenario. Usually a method
    \item \textbf{Test-suite}~-- a set of test-cases. Usually a class
  \end{itemize}
\end{frame}

% how test works?
\begin{frame}{The test-case workflow}
  How do tests work?
  \begin{itemize}
    \item Test-case started
    \item Test-case finished. Result: \textit{success} or \textit{failure}
    \item If result is a \textit{failure} then it generates a test-report
  \end{itemize}
\end{frame}

% why tests?
\begin{frame}{The purpose of the tests}
  Why do we need tests?
  \begin{itemize}
    \item Check the functionality. \textbf{Not the main purpose!}
    \item Fix the functionality. \textbf{The main purpose!!}
      \begin{itemize}
        \item To Simplify a refactoring
        \item Tests are also code specification
      \end{itemize}
    \item Unit-tests force you to write more modular code (according to TDD)
  \end{itemize}
\end{frame}

% good tests
\begin{frame}{Quality tests}
  Features of the quality test:
  \begin{itemize}
    \item Problem localiztion
    \item Stability
    \item Readable reports
    \item No duplicates \textbf{More tests $\ne$ better!}
    \item Tests should be simple. \textbf{We don't want to test the test-cases!}
    \pause
      \begin{itemize}
        \item Low \textit{cyclomatic complexity} of the test code
        \item Of course, other good coding practices
      \end{itemize}
  \end{itemize}
\end{frame}

% kind of test cases
\begin{frame}{Kinds of test-cases}
  \begin{itemize}
    \item Simple one-assertion
    \item General one-assertion
    \item Multi-assertion
    \item Property-base
    \item Parameterized
  \end{itemize}
\end{frame}

% simple one
\begin{frame}[fragile]{Simple one-assertion test-case}
  \begin{itemize}
    \item \textbf{Actual result}~-- the real result returning by the program
    \item \textbf{Expected result}~-- the result you are expecting
  \end{itemize}
  \begin{lstlisting}[style=scala]
    val actualResult: R = getActualResult()
    val expectedResult: R = getExpecatedResult()
    actualResult ==? expectedResult // is equal to?
  \end{lstlisting}
  \begin{itemize}
    \item The most simple kind of test-cases
    \item Prefer this kind if you can
  \end{itemize}
\end{frame}

% general one
\begin{frame}[fragile]{General one-assertion test-case}
  \begin{lstlisting}[style=scala]
    val actualResult: R = getActual()
    val expectedPredicate: (R => Boolean) = getExpecated()
    expectedPredicate(actualResult) ==? true 
  \end{lstlisting}
  \begin{itemize}
    \item The generalization of the \textit{Simple one-assertion test-case}
    \item Use \underline{Matcher} pattern to improve the quality of the test.
  \end{itemize}
\end{frame}

% multi
\begin{frame}[fragile]{Multi-assertion test-case}
  \begin{lstlisting}[style=scala]
    // ...
    actualResult1 ==? expectedResult1
    // ...
    actualResult2 ==? expectedResult3
    // ...
  \end{lstlisting}
  \begin{itemize}
    \item The generalization of the \textit{General one-assertion test-case}
    \item Avoid this kind of test-cases if you can.
    \item Use \underline{SoftAssert} pattern to imporve the quality of the test.
  \end{itemize}
\end{frame}

% property-base
\begin{frame}[fragile]{Property-base test-case}
  \begin{itemize}
    \item \textbf{Invariant}~-- the predicate that always must be true: $\forall x \in X: predicate(x)$
    \item This kind of test-cases checks the invariants
    \item There are some situations where this type of test is better than a usual test
    \item Use \underline{Generator} pattern to implement test-cases of this type.
  \end{itemize}
\end{frame}

% parameterized
\begin{frame}[fragile]{Multi-assertion test-case}
  \begin{lstlisting}[style=scala]
  def testCase(params: P) = {
    // ... scenario
  }
  \end{lstlisting}
  \begin{itemize}
    \item Before test-case was a function \textit{without} parameters
    \item Now test-case is function with \textit{with} parameters
    \item The generalization of \textit{any} kind of test-case.
  \end{itemize}
\end{frame}

% summary
\begin{frame}{Summary}
  What we have learnt:
  \begin{itemize}
    \item Tests should be as simple as possible
    \item Prefer Balck or Grey-box approach in most situations
    \item Choose the most appropriate kind of test-case
    \item Use test patterns like \textit{Matcher}, \textit{SoftAssert}, \textit{Generator}
  \end{itemize}
\end{frame}

% futher reading
\begin{frame}{For self-study}
  \begin{itemize}
    \item \textbf{Cyclomatic complexity}: use search engine
    \item \textbf{Test-driven-development (TDD)}: use search engine
    \item \textbf{\underline{Mock}} pattern
      \begin{itemize}
        \item Libraries: \textit{Mockito}, \textit{ScalaMock}
        \item Don't abuse it. Use it if needed only!
      \end{itemize}
  \end{itemize}
\end{frame}

% end
\begin{frame}{Questions}
  \centering\Large \textbf{Questions?}
\end{frame}

\end{document}

